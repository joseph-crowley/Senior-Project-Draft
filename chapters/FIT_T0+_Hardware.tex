
\chapter{The FIT T0+ (FT0) Detector}
The Fast Interaction Trigger is a detector system that measures the fundamental experimental parameters for analyzing particle interactions in ALICE: FIT measures the location and time of interactions, as well as multiplicity and centrality. These parameters are measured using scintillators and Cherenkov detectors. The logo for the FIT detector is shown, with its three subsystems, the Forward Diffraction Detector (FDD), the FV0, which measures centrality, and the FT0, in Figure \ref{fig:FIT_Logo}.  FIT T0+ (FT0) is a Cherenkov detector composed of quartz crystal radiators and photomultipliers; it measures the location and energy of particles that pass through the crystals. When signals are detected on multiple channels of the detector within a specified time interval, they trigger a signal that initiates data acquisition on other detector systems. This is called a coincidence measurement, and is the first data required in event reconstruction \cite{ALICE_Tracking_Readout_TDR}.

\begin{figure}[H]
    \centering
    \includegraphics[width=0.6\textwidth]{figures/FIT/FIT_Logo.jpg}
    \caption{Logo for the Fast Interaction Trigger, showing the FDD, FT0, and FV0 detectors.}
    \label{fig:FIT_Logo}
\end{figure}


\section{FT0 Hardware}
T0+ is a Cherenkov detector that consists of two sub-detectors, the A-side (FT0-A) and C-side (FT0-C). These detectors sit on opposite ends of the interaction point to increase the precision of measurements on interactions that have low transverse momenta. FT0-A, shown in Figure \ref{fig:my_label}, has twenty-four photomultipliers, each with four separate channels. FT0-C has twenty-eight photomultipliers, which are angled toward the beamline. Together, these Cherenkov detectors are used to measure important beam characteristics and trigger DAQ on other detectors in ALICE. The full FIT detector layout is shown in Figure \ref{fig:FIT_Layout}.

\begin{figure}[H]
    \centering
    \includegraphics[width=0.4\textwidth]{figures/FIT/FT0A_Structure.png}
    \caption{Structure of the FIT T0+ A-side, with 96 channels across 24 detector modules \cite{Slupecki}.}
    \label{fig:my_label}
\end{figure}

\begin{figure}[H]
    \centering
    \includegraphics[width=0.6\textwidth]{figures/FIT/FIT_Layout.png}
    \caption{The full FIT detector shown in schematic view, including FT0, FV0, and FDD. A-side and C-side are on opposite sides of the nominal interaction point, indicated by the tiny explosion \cite{Contreras}.}
    \label{fig:FIT_Layout}
\end{figure}


\subsection{Micro-Channel Plate Photomultiplier Tubes}
FT0 measures the location and energy of particles using a Cherenkov radiative quartz material coupled to a photomultiplier (PMT). Fast moving particles produce Cherenkov radiation within the quartz, which then hits the photocathode of the PMT. In a traditional PMT, the  photoelectrons ejected from the photocathode are accelerated toward metal plates called dynodes, which release more electrons. The acceleration and electron amplification is repeated across high potential differences until there are enough electrons to measure with a picoammeter. Because of this acceleration and amplification process, PMTs require high-voltage, on the order of kV. Micro-Channel Plate Photomultipliers (MCP-PMT) differ from regular photomultipliers in that they do not have discrete dynodes, instead having thin glass plates with many holes with diameter 1-100$\mu$m. MCP-PMTs are required in ALICE because of their minimal gain loss in strong magnetic fields. Having multiple anodes that read the current passing through provides better spatial resolution, which is why the FT0 MCP-PMT modules are each subdivided into four smaller area channels.  The electronics for the FIT T0 MCP-PMTs are shown in Fig. \ref{fig:MCP_PMT_Schematic}.
\begin{figure}[H]
    \centering
    \includegraphics[width=0.8\textwidth]{figures/FIT/PMT_Schematic.jpg} 
    \caption{Schematic for the MCP-PMTs used for the FIT T0+ Cherenkov detector \cite{Yury_MCP-PMT}.}
    \label{fig:MCP_PMT_Schematic}
\end{figure}



\subsection{High-Voltage Power Electronics for MCP-PMTs}{HV PMT Electronics}
MCP-PMTs require large potential differences to create a measurable current from a photon that hits the photocathode. However, if the voltage is too high, then the signal can become saturated, and the resolution of the detector is compromised. Fig. \ref{fig:MCP_PMT_Schematic} shows a schematic view of the MCP-PMTs in FIT. Each PMT is equipped with a voltage divider circuit that takes in a voltage of $\sim$2 kV, and supplies voltage to three of the elements in board one: the photocathode, and both sides of the micro-channel plate. The assembly of the HV power electronics was done in summer of 2019 at the FIT lab at CERN. This includes adding coaxial LEMO connectors to the HV cables and securing the cables to the power board (Board 3 in Fig. \ref{fig:MCP_PMT_Schematic}). The assembled power electronics are shown in Fig. \ref{fig:power_board}.

\begin{figure}[H]
    \centering
    \includegraphics[width=0.6\textwidth]{figures/FIT/power_board.jpg}
    \caption{HV power electronics for MCP-PMT in FT0-A.}
    \label{fig:power_board}
\end{figure}

\section{MCP-PMT Testing at CERN}\label{pmt_testing}
During LS2, the FIT collaboration has been working on the upgrades to the hardware for FT0, investigating the resolution, crosstalk, and trigger efficiency of the detector. The MCP-PMTs shown in Fig \ref{fig:MCP_PMT_Schematic}, which are split into four channels, must be tested for optimal performance. To test the most important aspects of MCP-PMT performance, laser pulses are sent to the MCPs to measure the response. The pulses are produced by a laser, then split to have a reference signal. The reference signal is measured by a traditional PMT with well known characteristics. The test signal is sent to one of the FT0 MCPs. By comparing the output current spikes from the reference and the test MCPs, we can test the performance of the MCP-PMTs as they are shipped from Photonis. A photo of the testing apparatus for individual MCPs is shown in Figure \ref{fig:PMT_Testing_Apparatus} while two different views of the the FT0-C detector test assembly are shown in Figure \ref{fig:FT0C_TestAssemblyFigs}. Oscilloscope traces from the reference PMT and an MCP being tested are shown in Figure \ref{fig:scope-trace}.


\begin{figure}[H]
    \centering
    \includegraphics[width=0.8\textwidth]{figures/FIT/PMT_Test_apparatus.jpg}
    \caption{MCP-PMT testing apparatus at the ALICE FIT Lab. Laser pulser, HV power supply, oscilloscope, function generator, isolating aluminum frame for light-tight data collection.}
    \label{fig:PMT_Testing_Apparatus}
\end{figure}

\begin{figure}[H]
  \centering
  \subfloat[FT0-C test assembly for MCP-PMTs in the FIT Lab.]{\includegraphics[width=0.4\textwidth]{figures/FIT/FT0_C_TestBench.jpg}\label{fig:FT0C_TestAssembly}}
  \hfill
  \subfloat[Side view of FT0-C test assembly, showing the Aluminum support structure that holds the quartz radiators and MCP-PMTs.]{\includegraphics[width=0.4\textwidth]{figures/FIT/FT0_C_half.jpg}\label{fig:FT0C_In_Al_Structure}}
  \caption{FT0-C in the FIT lab for MCP-PMT testing.}
  \label{fig:FT0C_TestAssemblyFigs}
\end{figure}


\begin{figure}[H]
    \centering
    \includegraphics[width=0.8\textwidth]{figures/FIT/PMT_test_pulse.jpg}
    \caption{Oscilloscope trace of a current spike from a reference MCP (yellow) and the test MCP (pink).}
    \label{fig:scope-trace}
\end{figure}

